\documentclass[12pt,letterpaper,titlepage,oneside]{book}
\usepackage{times}
\usepackage[utf8]{inputenc}
\usepackage[spanish]{babel}
\usepackage{amsmath}
\usepackage{amsfonts}
\usepackage{amssymb}
\usepackage{graphicx}
\usepackage{hyperref}
\usepackage{anysize} %Margenes
\marginsize{3cm}{2.5cm}{3cm}{3cm} %{Izq}{Der}{Sup}{Inf}
\usepackage{setspace}

\onehalfspacing
\setlength{\parskip}{3mm}
\setlength{\parindent}{0pt}

\begin{document}
\begin{titlepage}
\begin{center}
{\LARGE {\textsc{Universidad Mayor, Real y Pontifícia de \\ San Francisco Xavier de Chuquisaca}}} \\ [1.3cm]
{\Large \textsc \textbf{\textsc{Vicerrectorado}}} \\ [1.3cm]
{\Large \textsc{Centro de Estudios de Posgrado e Investigación}} \\ [1.3cm]
\includegraphics[width=4.4cm]{escudo.jpg} \\ [1.3cm]
{\Large {\textsc{Juego de rol para la enseñanza de programación}}} \\[1.3cm]
 {Trabajo en opción al grao de Doctor en Ciencias de la Educación} \\[1.3cm]
{Autor: \textsc{José Boris Bellido Santa María}}\\[1.3cm]
Sucre, Septiembre de 2015
\end{center}
\end{titlepage}
\pagenumbering{roman}

\chapter*{Cesión de Derechos}
Al presentar este trabajo como requisito previo para la obtención del Título de Magister en
Educación Superior de la Universidad Mayor, Real y Pontificia de San Francisco Xavier de Chuquisaca, autorizo al Centro de Estudios de Posgrado e Investigación o a la Biblioteca de la Universidad, para que se haga de este trabajo un documento disponible para su lectura, según normas de la Universidad.

También cedo a la Universidad Mayor, Real y Pontificia de San Francisco Xavier de Chuquisaca, los derechos de publicación de este trabajo o parte de él, manteniendo mis derechos de autor hasta un periodo de 30 meses posterior a su aprobación.
\\[3cm]

\begin{center}
José Boris Bellido Santa María
\end{center} 

\chapter*{Dedicatoria}
\begin{center}
Para mis padres, \\
José y Yolanda.
\end{center} 

\chapter*{Agradecimientos}
%TODO

\tableofcontents
\markboth{Índice general}{Índice general}
\listoftables
\markboth{Índice de tablas}{Índice de tablas}
\listoffigures
\markboth{Índice de figuras}{Índice de figuras}

\chapter*{Resumen}
\markboth{Resumen}{Resumen}
%TODO
\frontmatter

\mainmatter

\chapter*{Introducción}
\addcontentsline{toc}{chapter}{Introducción}
\markboth{Introducción}{Introducción}
\section*{Antecedentes}
\addcontentsline{toc}{section}{Antecedentes}

\section*{Situación problémica}
\addcontentsline{toc}{section}{Situación problémica}

\section*{Formulación del problema de investigación}
\addcontentsline{toc}{section}{Formulación del problema de investigación}
 (Como afirmación  o alternativamente como  pregunta)

\section*{Hipótesis}
\addcontentsline{toc}{section}{Hipótesis}

\subsection*{Conceptualización de las variables}
\addcontentsline{toc}{subsection}{Conceptualización de las variables}

\subsection*{Operacionalización las de variables}
\addcontentsline{toc}{subsection}{Operacionalización de las variables}
( Cuadro que determina por variable, sus dimensiones, categorías e indicadores)

\section*{Objetivos de la investigación}
\addcontentsline{toc}{section}{Objetivos de la investigación}

\subsection*{Objetivo general}
\addcontentsline{toc}{subsection}{Objetivo general}

\subsection*{Objetivos específicos}
\addcontentsline{toc}{subsection}{Objetivos específicos}

\section*{Justificación}
\addcontentsline{toc}{section}{Justificación}
  (Relevancia y    pertinencia  social,  actualidad y novedad del tema)

\section*{Diseño metodológico}
\addcontentsline{toc}{section}{Diseño metodológico}
 .(Acorde al enfoque cuantitativo en las ciencias sociales)2

\chapter{Fundamentos teóricos y de contexto}
\markboth{Fundamentos teóricos y de contexto}{Fundamentos teóricos y de contexto}

\chapter{Análisis e interpretación de las indagaciones de campo}
\markboth{Análisis e interpretación de las indagaciones de campo}{Análisis e interpretación de las indagaciones de campo}

\chapter{Propuesta y validación}
\markboth{Propuesta y validación}{Propuesta y validación}

\chapter*{Conclusiones y recomendaciones}
\addcontentsline{toc}{chapter}{Conclusiones y recomendaciones}
\markboth{Conclusiones y recomendaciones}{Conclusiones y recomendaciones}
\end{document}